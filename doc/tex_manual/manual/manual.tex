%%%%%%%%%%%%%%%%%%%%%%%%%%%%%%%%%%%%%%%%%
% The Legrand Orange Book
% LaTeX Template
% Version 2.1 (14/11/15)
%
% This template has been downloaded from:
% http://www.LaTeXTemplates.com
%
% Mathias Legrand (legrand.mathias@gmail.com) with modifications by:
% Vel (vel@latextemplates.com)
%
% License:
% CC BY-NC-SA 3.0 (http://creativecommons.org/licenses/by-nc-sa/3.0/)
%
% Compiling this template:
% This template uses biber for its bibliography and makeindex for its index.
% When you first open the template, compile it from the command line with the 
% commands below to make sure your LaTeX distribution is configured correctly:
%
% 1) pdflatex main
% 2) makeindex main.idx -s StyleInd.ist
% 3) biber main
% 4) pdflatex main x 2
%
% After this, when you wish to update the bibliography/index use the appropriate
% command above and make sure to compile with pdflatex several times 
% afterwards to propagate your changes to the document.
%
% This template also uses a number of packages which may need to be
% updated to the newest versions for the template to compile. It is strongly
% recommended you update your LaTeX distribution if you have any
% compilation errors.
%
% Important note:
% Chapter heading images should have a 2:1 width:height ratio,
% e.g. 920px width and 460px height.
%
%%%%%%%%%%%%%%%%%%%%%%%%%%%%%%%%%%%%%%%%%

%----------------------------------------------------------------------------------------
%	PACKAGES AND OTHER DOCUMENT CONFIGURATIONS
%----------------------------------------------------------------------------------------

\documentclass[11pt,fleqn]{book} % Default font size and left-justified equations

%----------------------------------------------------------------------------------------

\input{imports/structure} % Insert the commands.tex file which contains the majority of the structure behind the template
\input{imports/code_snippets}

\usepackage{draftwatermark}
\SetWatermarkText{Draft}
\SetWatermarkScale{0.8}
\SetWatermarkColor[rgb]{0.9,0.9,0.9}

\begin{document}

%----------------------------------------------------------------------------------------
%	TITLE PAGE
%----------------------------------------------------------------------------------------

\begingroup
\thispagestyle{empty}
\begin{tikzpicture}[remember picture,overlay]
\coordinate [below=12cm] (midpoint) at (current page.north);
\node at (current page.north west)
{\begin{tikzpicture}[remember picture,overlay]
\node[anchor=north west,inner sep=0pt] at (0,0) {\includegraphics[width=\paperwidth]{background}}; % Background image
\draw[anchor=north] (midpoint) node [fill=ocre!30!white,fill opacity=0.6,text opacity=1,inner sep=1cm]{\Huge\centering\bfseries\sffamily\parbox[c][][t]{\paperwidth}{\centering The Blastpit Manual\\[15pt] % Book title
%{\Large Network control of laser marking}\\[20pt] % Subtitle
{\huge R.F. Bevan \& Company Limited}}}; % Author name
\end{tikzpicture}};
\end{tikzpicture}
\vfill
\endgroup

%----------------------------------------------------------------------------------------
%	COPYRIGHT PAGE
%----------------------------------------------------------------------------------------

\newpage
~\vfill
\thispagestyle{empty}

\noindent Copyright \copyright\ 2016 R.F. Bevan \& Company Limited\\ % Copyright notice

%\noindent \textsc{Published by Publisher}\\ % Publisher

%\noindent \textsc{book-website.com}\\ % URL

\noindent This manual is based on the Legrand Orange Book LaTeX template, originally created by Mathias Legrand (legrand.mathias@gmail.com) with modifications by Vel (vel@latextemplates.com)\\

\noindent Licensed under the Creative Commons Attribution-NonCommercial 3.0 Unported License (the ``License''). You may not use this file except in compliance with the License. You may obtain a copy of the License at \url{http://creativecommons.org/licenses/by-nc/3.0}. Unless required by applicable law or agreed to in writing, software distributed under the License is distributed on an \textsc{``as is'' basis, without warranties or conditions of any kind}, either express or implied. See the License for the specific language governing permissions and limitations under the License.\\ % License information

%\noindent \textit{First printing, March 2013} % Printing/edition date

\newpage

This manual is being used for "Documentation Driven Development", and as such
is an expectation of how the finished software should behave. It may therefore
contain references to features that are not yet incorporated into the main
software package, and may never actually be developed.

\newpage

%----------------------------------------------------------------------------------------
%	TABLE OF CONTENTS
%----------------------------------------------------------------------------------------

\chapterimage{chapter_head_1.pdf} % Table of contents heading image

\pagestyle{empty} % No headers

\tableofcontents % Print the table of contents itself

\cleardoublepage % Forces the first chapter to start on an odd page so it's on the right

\pagestyle{fancy} % Print headers again

%----------------------------------------------------------------------------------------
%	PART
%----------------------------------------------------------------------------------------

\part{Introduction}

%----------------------------------------------------------------------------------------
%	CHAPTER 1
%----------------------------------------------------------------------------------------

\chapterimage{chapter_head_2.pdf} % Chapter heading image

\chapter{Introduction}

\section{What is blastpit?}\index{Blastpit (about)}

Blastpit is a free software package that provides network control of the Rofin
EasyJewel laser system, and allows for integration into other CAD or
illustration software. You can write programs in the popular Python language
that can control most aspects of the laser system, making it easy to automate
repetitive tasks.

\section{Features}\index{Features}

\begin{itemize}
        \item Network control of the laser functions
        \item Fully scriptable from the Python language
        \item 3d mouse and controller support
        \item Plugins for Inkscape, a vector graphics editor, and FreeCAD, a powerful CAD package
        %\item Utility functions to help cleanup drawings and prevent problems such as incorrect hatching
        %\item Incorporates image recognition for verification and error prevention
        %\item Improved bitmap support
        %\item Support for GNU Octave, a powerful numerical language with extensive geometry libraries
        \item Non-proprietary and free of digital restrictions
\end{itemize}

\section{Why use it?}

Blastpit aims to be a seamless interface between your chosen design software
and the laser hardware. If everything works as planned, you really shouldn't 
notice blastpit at all.

\section{An Example}

Here is an example of a Python script that marks a circle:

\insertcode{"code_snippets/circle.py"}{Simple Python program.} % The first argument is the script location/filename and the second is a caption for the listing

%

\part{Installation}

\chapter{Installation}

\section{Getting the software}

\section{Installing Blastpit}

\subsection{Installing the Inkscape plugin}\index{Inkscape, installing}\index{Plugin, Inkscape}

\subsection{Installing the FreeCAD workbench}

\section{Compiling it yourself}\index{Compiling}

%

\part{Using Blastpit}

\chapter{Using Blastpit}

\section{Using Blastpit with Inkscape}\index{Inkscape, using}

\section{Using Blastpit with FreeCAD}

\section{Using a 3d controller for real-time adjustment}

\section{Writing Python programs}\index{Python}

\section{Using GNU Octave}\index{Octave}\index{GNU Octave}

%

\part{Additional Features}

\chapter{Calibration}

When using the rotary axis to mark a continuous pattern, it may be necessary to digitise the item being engraved in order to avoid seams at the boundaries between rotations. By marking calibration lines during a dry run, it is possible to compensate for variations in the item diameter or axial parallelism.


\section{Performing a Calibrating Dry Run}\index{calibration, rotary axis}


%------------------------------------------------

%\section{Lists}\index{Lists}
%
%Lists are useful to present information in a concise and/or ordered way\footnote{Footnote example...}.
%
%\subsection{Numbered List}\index{Lists!Numbered List}
%
%\begin{enumerate}
%\item The first item
%\item The second item
%\item The third item
%\end{enumerate}
%
%\subsection{Bullet Points}\index{Lists!Bullet Points}
%
%\begin{itemize}
%\item The first item
%\item The second item
%\item The third item
%\end{itemize}
%
%\subsection{Descriptions and Definitions}\index{Lists!Descriptions and Definitions}
%
%\begin{description}
%\item[Name] Description
%\item[Word] Definition
%\item[Comment] Elaboration
%\end{description}
%
%%----------------------------------------------------------------------------------------
%%	CHAPTER 2
%%----------------------------------------------------------------------------------------
%
%\chapter{In-text Elements}
%
%\section{Theorems}\index{Theorems}
%
%This is an example of theorems.
%
%\subsection{Several equations}\index{Theorems!Several Equations}
%This is a theorem consisting of several equations.
%
%\begin{theorem}[Name of the theorem]
%In $E=\mathbb{R}^n$ all norms are equivalent. It has the properties:
%\begin{align}
%& \big| ||\mathbf{x}|| - ||\mathbf{y}|| \big|\leq || \mathbf{x}- \mathbf{y}||\\
%&  ||\sum_{i=1}^n\mathbf{x}_i||\leq \sum_{i=1}^n||\mathbf{x}_i||\quad\text{where $n$ is a finite integer}
%\end{align}
%\end{theorem}
%
%\subsection{Single Line}\index{Theorems!Single Line}
%This is a theorem consisting of just one line.
%
%\begin{theorem}
%A set $\mathcal{D}(G)$ in dense in $L^2(G)$, $|\cdot|_0$. 
%\end{theorem}
%
%%------------------------------------------------
%
%\section{Definitions}\index{Definitions}
%
%This is an example of a definition. A definition could be mathematical or it could define a concept.
%
%\begin{definition}[Definition name]
%Given a vector space $E$, a norm on $E$ is an application, denoted $||\cdot||$, $E$ in $\mathbb{R}^+=[0,+\infty[$ such that:
%\begin{align}
%& ||\mathbf{x}||=0\ \Rightarrow\ \mathbf{x}=\mathbf{0}\\
%& ||\lambda \mathbf{x}||=|\lambda|\cdot ||\mathbf{x}||\\
%& ||\mathbf{x}+\mathbf{y}||\leq ||\mathbf{x}||+||\mathbf{y}||
%\end{align}
%\end{definition}
%
%%------------------------------------------------
%
%\section{Notations}\index{Notations}
%
%\begin{notation}
%Given an open subset $G$ of $\mathbb{R}^n$, the set of functions $\varphi$ are:
%\begin{enumerate}
%\item Bounded support $G$;
%\item Infinitely differentiable;
%\end{enumerate}
%a vector space is denoted by $\mathcal{D}(G)$. 
%\end{notation}
%
%%------------------------------------------------
%
%\section{Remarks}\index{Remarks}
%
%This is an example of a remark.
%
%\begin{remark}
%The concepts presented here are now in conventional employment in mathematics. Vector spaces are taken over the field $\mathbb{K}=\mathbb{R}$, however, established properties are easily extended to $\mathbb{K}=\mathbb{C}$.
%\end{remark}
%
%%------------------------------------------------
%
%\section{Corollaries}\index{Corollaries}
%
%This is an example of a corollary.
%
%\begin{corollary}[Corollary name]
%The concepts presented here are now in conventional employment in mathematics. Vector spaces are taken over the field $\mathbb{K}=\mathbb{R}$, however, established properties are easily extended to $\mathbb{K}=\mathbb{C}$.
%\end{corollary}
%
%%------------------------------------------------
%
%\section{Propositions}\index{Propositions}
%
%This is an example of propositions.
%
%\subsection{Several equations}\index{Propositions!Several Equations}
%
%\begin{proposition}[Proposition name]
%It has the properties:
%\begin{align}
%& \big| ||\mathbf{x}|| - ||\mathbf{y}|| \big|\leq || \mathbf{x}- \mathbf{y}||\\
%&  ||\sum_{i=1}^n\mathbf{x}_i||\leq \sum_{i=1}^n||\mathbf{x}_i||\quad\text{where $n$ is a finite integer}
%\end{align}
%\end{proposition}
%
%\subsection{Single Line}\index{Propositions!Single Line}
%
%\begin{proposition} 
%Let $f,g\in L^2(G)$; if $\forall \varphi\in\mathcal{D}(G)$, $(f,\varphi)_0=(g,\varphi)_0$ then $f = g$. 
%\end{proposition}
%
%%------------------------------------------------
%
%\section{Examples}\index{Examples}
%
%This is an example of examples.
%
%\subsection{Equation and Text}\index{Examples!Equation and Text}
%
%\begin{example}
%Let $G=\{x\in\mathbb{R}^2:|x|<3\}$ and denoted by: $x^0=(1,1)$; consider the function:
%\begin{equation}
%f(x)=\left\{\begin{aligned} & \mathrm{e}^{|x|} & & \text{si $|x-x^0|\leq 1/2$}\\
%& 0 & & \text{si $|x-x^0|> 1/2$}\end{aligned}\right.
%\end{equation}
%The function $f$ has bounded support, we can take $A=\{x\in\mathbb{R}^2:|x-x^0|\leq 1/2+\epsilon\}$ for all $\epsilon\in\intoo{0}{5/2-\sqrt{2}}$.
%\end{example}
%
%\subsection{Paragraph of Text}\index{Examples!Paragraph of Text}
%
%\begin{example}[Example name]
%\lipsum[2]
%\end{example}
%
%%------------------------------------------------
%
%\section{Exercises}\index{Exercises}
%
%This is an example of an exercise.
%
%\begin{exercise}
%This is a good place to ask a question to test learning progress or further cement ideas into students' minds.
%\end{exercise}
%
%%------------------------------------------------
%
%\section{Problems}\index{Problems}
%
%\begin{problem}
%What is the average airspeed velocity of an unladen swallow?
%\end{problem}
%
%%------------------------------------------------
%
%\section{Vocabulary}\index{Vocabulary}
%
%Define a word to improve a students' vocabulary.
%
%\begin{vocabulary}[Word]
%Definition of word.
%\end{vocabulary}
%
%%----------------------------------------------------------------------------------------
%%	PART
%%----------------------------------------------------------------------------------------
%
%\part{Part Two}
%
%%----------------------------------------------------------------------------------------
%%	CHAPTER 3
%%----------------------------------------------------------------------------------------
%
%\chapterimage{chapter_head_1.pdf} % Chapter heading image
%
%\chapter{Presenting Information}
%
%\section{Table}\index{Table}
%
%\begin{table}[h]
%\centering
%\begin{tabular}{l l l}
%\toprule
%\textbf{Treatments} & \textbf{Response 1} & \textbf{Response 2}\\
%\midrule
%Treatment 1 & 0.0003262 & 0.562 \\
%Treatment 2 & 0.0015681 & 0.910 \\
%Treatment 3 & 0.0009271 & 0.296 \\
%\bottomrule
%\end{tabular}
%\caption{Table caption}
%\end{table}
%
%%------------------------------------------------
%
%\section{Figure}\index{Figure}
%
%\begin{figure}[h]
%\centering\includegraphics[scale=0.5]{placeholder}
%\caption{Figure caption}
%\end{figure}
%
%%----------------------------------------------------------------------------------------
%%	BIBLIOGRAPHY
%%----------------------------------------------------------------------------------------
%
%%\chapter*{Bibliography}
%%\addcontentsline{toc}{chapter}{\textcolor{ocre}{Bibliography}}
%%\section*{Books}
%%\addcontentsline{toc}{section}{Books}
%%\printbibliography[heading=bibempty,type=book]
%%\section*{Articles}
%%\addcontentsline{toc}{section}{Articles}
%%\printbibliography[heading=bibempty,type=article]
%
%%----------------------------------------------------------------------------------------
%%	INDEX
%%----------------------------------------------------------------------------------------

\cleardoublepage
\phantomsection
\setlength{\columnsep}{0.75cm}
\addcontentsline{toc}{chapter}{\textcolor{ocre}{Index}}
\printindex

%----------------------------------------------------------------------------------------

\end{document}
